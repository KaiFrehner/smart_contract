\documentclass[a4paper,titlepage]{article}
\usepackage{float}
\usepackage[utf8]{inputenc}
\usepackage[english]{babel}
\usepackage[T1]{fontenc}
\usepackage{geometry}
\geometry{a4paper,left=35mm,right=35mm,top=4cm,bottom=4cm}
\usepackage{graphicx}
\usepackage{amsmath,amsfonts,amssymb}
\usepackage{multirow}
\numberwithin{equation}{section} %Nummeriert mathematische Umgebungen nach sections durch
\usepackage[nodisplayskipstretch]{setspace} %lässt Abstand zwischen align-Umgebungen und Text auf 1.0
\setstretch{1}% ergibt 1,5-fachen Zeilenabstand
\setlength{\parindent}{0pt} % kein indent bei absätzen
\setlength{\parskip}{\baselineskip} % abstand zwischen absätzen
\linespread{0.95} % controls the tightness overall
\usepackage{hyperref }
\usepackage{nameref}
\usepackage{booktabs}

%%%%%%%%%%%%%%%%%%%%%%%%%%%%%%%%%%%%%%%%%%%%%%%%%%%%%%%%%%%%%%%%%%%%%%%%%%%%%%%%%%%%%%%%%%%%%%%%
%%%%%%%%%%%%%%%%%%%%%%%%%%%%%%%%%%%%%%%%%%%%%%%%%%%%%%%%%%%%%%%%%%%%%%%%%%%%%%%%%%%%%%%%%%%%%%%%

\begin{document}
\begin{titlepage}

\begin{center}
	\includegraphics[scale=1.5]{Logo_Basel}
\end{center}
\vspace{2.5cm}
\begin{center}                     
				{\Large\scshape Smart Contracts and Decentralized Finance}\\*[5mm]
        {\bf\Large\scshape Parametric Weather Insurance}\\*[12mm]         
\end{center}  
\vspace{4,5cm}
\begin{tabbing}
        submitted by:		\hspace{6.5cm}\=			supervised by:\\*[2mm]
       \bf{Kai Frehner, Benjamin Gerber} 								\> \bf{Prof. Dr. Fabian Schär}\\*[2mm]		
        Matrikel--Nr.:  20-647-970, 21-550-579\\*[2mm]										\> Place, Date: \\*[2mm]		   							\> \textbf{Basel, 21.11.2025}   \\*[2mm]
\end{tabbing}
\end{titlepage}
\newpage

%%%%%%%%%%%%%%%%%%%%%%%%%%%%%%%%%%%%%%%%%%%%%%%%%%%%%%%%%
%%%%%%%%%%%%%%%%%%%%%%%%%%%%%%%%%%%%%%%%%%%%%%%%%%%%%%%%%

%\thispagestyle{empty}
\pagenumbering{Roman}
\setcounter{page}{2}


%%%%%%%%%%%%%%%%%%%%%%%%%%%%%%%%%%%%%%%%%%%%%%%%%%%%%%%%%
%%%%%%%%%%%%%%%%%%%%%%%%%%%%%%%%%%%%%%%%%%%%%%%%%%%%%%%%%

\tableofcontents
\newpage

%%%%%%%%%%%%%%%%%%%%%%%%%%%%%%%%%%%%%%%%%%%%%%%%%%%%%%%%%
%%%%%%%%%%%%%%%%%%%%%%%%%%%%%%%%%%%%%%%%%%%%%%%%%%%%%%%%%

\pagenumbering{arabic}
\setcounter{page}{1}
\section{Introduction}
Parametric insurance has become one of the most dynamic innovations in the insurance industry. Unlike traditional indemnity products, where the payout depends on an assessed loss, parametric insurance compensates the policyholder as soon as a predefined, objective trigger is reached. This structure eliminates lengthy loss adjustment processes, increases transparency, and allows for flexible product design across various lines of business such as agriculture, natural catastrophe, and travel insurance.

A major technological driver behind the rise of parametric solutions is the use of smart contracts. As self-executing programs on a blockchain, smart contracts automatically enforce contractual rules once the relevant conditions are fulfilled. Their deterministic execution, immutability, and auditability make them well-suited for parametric insurance, where payouts rely on measurable external data such as precipitation, wind speed, or flight delays. By integrating smart contracts with reliable oracle systems, the entire claims process, from data retrieval to payout, can be automated. This reduces administrative effort, prevents disputes, and enables fast, predictable settlements for policyholders.

This paper outlines the core principles of parametric insurance and contrasts them with traditional indemnity structures. We then present a real-world example from Baloise’s multi-parametric travel insurance product, focusing on the bad-weather component developed with the German start-up Wetterheld. The central part of the project is a simplified implementation of a parametric weather insurance product using a smart contract on the Ethereum Sepolia test network, illustrating how blockchain technology can enhance transparency, automation, and efficiency in parametric insurance.
\newpage

%%%%%%%%%%%%%%%%%%%%%%%%%%%%%%%%%%%%%%%%%%%%%%%%%%%%%%%%%
%%%%%%%%%%%%%%%%%%%%%%%%%%%%%%%%%%%%%%%%%%%%%%%%%%%%%%%%%



\section{Parametric Insurance}
Thinking about nonlife insurance the intuitive idea is that a predefined risk is covered by an insurance company, which, in occurrence of a loss or damage, would pay out a certain amount to the claimant. This amount would have a direct relation to the insured loss. In the world of nonlife insurance solutions there are far more classes of contracts than this classical kind of indemnity insurance. The figure below is showing the different kinds of indemnity and non-indemnity insurance contracts:

\begin{figure}[h] % [h] = "here", direkt an dieser Stelle
    \centering
    \includegraphics[width=0.5\textwidth]{nonlifecontracts} % Dateiname des Bildes
    \caption{Classification of nonlife contracts \cite{LinKwon}}
    \label{fig:beispiel}
\end{figure}
In the following we will refer to "Pure Parametric Contracts" using the term "Parametric Insurance". This is the class of contracts, were a predefined amount is paid to the insured, if a specific trigger is hit, regardless of the loss amount. Examples for such triggers could be the magnitude of an earthquake, wind speed or the delay of a plane. The following table highlights differences between traditional indemnity insurance and parametric solutions:

\begin{table}[H] 
    \centering
    \begin{tabular}{|p{2.25cm}|p{5cm}|p{5cm}|}
        \hline
        \toprule
         & \textbf{Traditional insurance }& \textbf{Parametric insurance} \\
        \midrule \hline\hline
        Recovery & Reimbursement of ajusted loss & Pre-agreed payment structure based on a event parameter (often binary) \\\hline
        Trigger & Payment triggered by actual loss or damage & Payment triggered by an event occurrence exceeding the parametric threshold \\\hline
        Loss assement and payment & Complex and based on loss adjuster assesment & Transparent, predictable, based on a parameter, quick settlement \\\hline
        Structure & Standard product and contract wordings & Tailored product with high structuring flexibility \\
        \bottomrule
        \hline
    \end{tabular}
    \caption{Traditional versus parametric insurance  \cite{ParamGuide}}
    \label{tab:example}
\end{table}

Those differences in the structure mentioned in Table 1 makes parametric solutions attractive for different Lines of Businesses such as Agriculture, Nature Catastrophe or Travel Insurance \cite{ParamGuide}. In recent years it is one of the fastest growing areas in the world of insurance solutions expecting to triple in premium income within the 2020s to almost 30 billion USD \cite{SwissReParamIns}. Still a niche product compared to the seven trillion USD insurance premium income worldwide \cite{AllianzReport}. 

\subsection{Baloise and Wetterheld - A parametric travel insurance}

In 2023  Swiss insurer Baloise achieved second place at the Swiss Insurance Innovation award prize for innovation for its multi-parametric travel insurance solution "Parasurance". With Parasurance customers can decide whether they cover for flight delay, luggage delay, bad weather or a combination of those on their trip. Further, a motor third party liability insurance can be added to the coverage. The parametric bad weather insurance is a collaboration with the German start-up Wetterheld. The trigger in this parametric insurance, is hit, if the precipitation at a fixed destination, exceeds a certain amount (in mm) within a time period. While Baloise provides the product, Wetterheld is responsible for the tracking of the parameters. The scope of the insurance provider limits as follows:
% Kommentar: ergänzen, Wetterheld/Baloise versichert mehrere Risiken. Erwähnen, dass wir nur den Wetterteil anschauen.

\paragraph*{Time restriction}
To prevent moral hazard, parametric weather insurance cover cannot be bought later than 14 days before the trip starts. The accuracy of weather forecasts has been shown to decrease exponentially, with forecasts beyond two weeks being no more reliable than random guessing. The available duration of a trip covered by bad-weather insurance is 2-92 days.

\paragraph*{Geographical scope}
The insured has to provide GPS coordinates of his trip destination. Wetterheld then uses publicly accessible weather data from meteostat to verify whether the policyholder is entitled to compensation \cite{meteostat}, checking the precipitation data at the reported coordinates. Change of place due to travelling is not possible in the bad weather insurance. Only trips within Europe can be insured by Wetterheld.

\paragraph*{Definition of bad weather}
The trigger for this parametric insurance is hit, if the precipitation (rainfall, snow, hail etc.) exceeds 2.9mm within one day only considering the time 10 a.m. until 6 p.m.A rain amount of 2.9mm is approximately equivalent to 90 minutes of light rain or 10-15 min of heavy rain \cite{dwd}. 

\paragraph*{Deductible days}
The bad weather coverage also considers the destination of a trip in terms of the amount of precipitation expected in this area. The probability for a wet trip is not the same when travelling to the United Kingdom as for travelling to Andalusia. Therefore Wetterheld introduced what can be understood as deductible days. For example for a seven days trip to Cardiff the first three days may not be covered, while on the Sevilla trip only the first day is not covered due to different expected precipitation frequencies. Furthermore, the number of days with significant precipitation varies considerably throughout the calendar year.

\paragraph*{Pricing}
Calculating a reasonable risk premium is the most crucial part and remains confidential. No information about the modelling and pricing was available, the only transparent part of a parametric insurance is the pay-out trigger.

\section{Implementation of a parametric weather insurance on a Smart Contract}
In this section the implementation of a parametric weather insurance on a Smart Contract is discussed. First a brief summary of Smart Contracts is provided.

\subsection{Basics of Smart Contracts}
% ChatGPT:  Bitte eine Halbe Seite über Smart Contracts schreiben.

A smart contract is a self-executing program stored on a blockchain that automatically enforces and carries out predefined rules once specific conditions are met. Its logic is executed deterministically by all validating nodes in the network, ensuring identical outcomes without relying on intermediaries or trust between contracting parties. Because all state transitions and transactions are recorded on an immutable distributed ledger, smart contracts offer transparency, auditability, and resistance to tampering.

Technically, a smart contract is deployed as bytecode at a dedicated blockchain address and can be invoked through transactions that trigger state changes. The contract used in this study is implemented in Solidity and deployed on the \textit{Sepolia} Ethereum test network, a low-cost proof-of-stake testnet that reproduces the execution environment of the Ethereum mainnet. This allows realistic testing of contract logic—such as fund transfers, threshold evaluations, and automated payouts—without financial exposure.

Smart contracts are frequently used to automate processes in decentralized applications, including parametric insurance. Because blockchains cannot natively access external data, such contracts rely on oracle mechanisms that deliver off-chain information (e.g., weather measurements) to the chain. Once this data is received, the contract evaluates it according to predefined criteria and deterministically executes the corresponding outcome, such as triggering a payout.


\subsection{Simplified Implementation of parametric weather insurance}
For the project in "Smart Contracts and Decentralised Finance" the product presented was partially implemented in a smart contract on the Sepolia-ETH test net. The code was programmed using the programming language Solidity and the development environment used was the REMIX browser IDE. Many aspects were simplified and some were out of the scope of our project. When a topic was considered out of scope, the theoretical solution is outlines as well as the simplified implemented version are specified.

\paragraph*{Time restriction}
The time restrictions implemented in the smart contract mirror those used in the Baloise insurance product. The start and end dates of a policy are validated using Solidity's \texttt{require} statements. For testing purposes, the minimum waiting period of 14 days has been removed.


\paragraph*{Geographical scope}
In this implementation the insured would provide a start and an end date of his trip as well as gps coordinates of his location. The location is then used to get an estimate for the number of days with precipitation larger than 2.9mm (details see below "Deductible days") using R. The geographical scope of the program is Europe (incl. Iceland, Scandinavia and Turkey). The % Kommentar: Hier ist der Satz nicht fertig. 
% Kommentar: entweder die Analyse der Daten einbringen, 
% Kommentar: oder einfach sagen es war out of scope. 

\paragraph*{Definition of bad weather}
Bad weather is defined by 2.9mm precipitation within a time period between 8 AM and 6 PM per day in accordance with the Baloise product.

\paragraph*{Deductible days}
The calculation which lies behind the number of deductible days from Wetterheld is not transparently available for customers. For this project we had our own approach using five risk categories (A-E) to find the number of deductible days for a certain location. Copernicus Climate Change services is a program by the European Comission that provides historical weather data for drought, flood, temperature, precipitation etc. We used historical precipitation data (1983-2023) which is obtained through satellite imagery for each 0.5° x 0.5° grid cell. Each of the five risk category represents a 20-percent quantile of the annual precipitation amounts. For exmaple if the average annual precipitation in our location is in the 35th percentile of our dataset, its risk category would be "B". The number of deductibles is then determined by the formula:
\[
D_{\text{ded}} =  \operatorname{round}\!\big(D_{\text{tot}} \cdot \mu_X\big)
\]

where: $D_{\text{ded}} \text{ = Deductible Days}, D_{\text{tot}} \text{ = Duration of the trip}, 
\mu_X \text{ = risk factor, with } \mu_A = 0.1, \mu_B = 0.2, \dots, \mu_E = 0.5$


\paragraph*{Pricing}
Due to the proprietary nature of the Baloise product, the exact pricing methodology is not publicly disclosed and is therefore considered out of scope for this project. For testing purposes, a simplified and deliberately rudimentary premium formula is employed. This approximation derives the premium from the \textbf{duration} of the insured trip, the level of \textbf{coverage}, the number of \textbf{deductible days}, and a coarse estimate of the \textbf{likelihood of rain in the insured region}. In addition, a \textbf{safety margin} of $\Phi = \tfrac{11}{10}$ is included to account for uncertainty.

\[
\text{premium}
= \text{coverage} \times (\text{duration} - \text{thresholdForRegion}) \times \mu_X \times \Phi
\]


\paragraph*{Example Berlin in September 2025}
In order to illustrate a parametric weather insurance, we assume a trip to Berlin between 9~September and 17~September~2025 (days 252--260 in Figure~\ref{fig:bsp_trigger}). During this period, precipitation exceeds the threshold of 2.9\,mm on four days. Berlin is assigned to risk category ``B''. Therefore, the number of deductible days is
\[
D_{\text{ded}} = \operatorname{round}\!\big(8 \cdot 0.2\big) = 2.
\]
Consequently, the policy would have triggered a payout of
\[
2 \times \text{coverage}.
\]

\begin{figure}[h] % [h] = "here", direkt an dieser Stelle
    \centering
    \includegraphics[width=0.5\textwidth]{weather_2025_Sep} % Dateiname des Bildes
    \caption{example of a triggered parametric weather insurance}
    \label{fig:bsp_trigger}
\end{figure}
\newpage
\subsection{Specifics of the Smart Contract}

This section describes the implementation of the parametric weather insurance using a smart contract. Some features outlined above are simplified, either because they fall outside the scope of this project or because they proved impractical for testing purposes.

The basic structure of the smart contract is that the insurer deploys the contract, and users can purchase policies directly on-chain. Policy information (i.e., policyholder, region of travel, number of deductible days, and the threshold for bad weather (2.9\,mm)) is stored within the contract. Aggregated weather data for the relevant time period is imported once daily, which triggers the function that checks which policies are affected. This function is described in more detail below.

\paragraph*{Input Checks}
Because the contract requires user interaction when setting up a policy, all inputs undergo validation. This is primarily achieved using Solidity's \texttt{require} statements, which ensure that user-provided data is valid and that the premium payment meets the required amount.

\paragraph*{Events}
Key state changes in the smart contract are recorded using events. These events are written to the transaction logs, providing an auditable record of contract activity. In this implementation, events are emitted when a new policy is created, when new weather data is imported, and when policies are affected by an incoming weather data point. Additionally, the contract emits an event when a policy expires and a payout is executed.

\paragraph*{Data Import}
To supply the contract with external information, an oracle mechanism is used. A blockchain oracle is an external service that delivers off-chain data—such as weather measurements—to a smart contract, enabling it to react to real-world events that the blockchain cannot observe natively. As noted by Antonopoulos and Wood (2018), oracles are essential for linking deterministic on-chain logic with unpredictable off-chain inputs. In this implementation, weather data is imported manually through a function call, which enables rapid testing by allowing new data points to be added within seconds rather than waiting for real-world observations.

\paragraph*{Outline of the \texttt{trigger} Function} \label{sec:trigger}
The core logic of the parametric insurance mechanism is implemented in the \texttt{trigger} function, which is executed whenever a new weather data point is uploaded. Each time the function runs, it iterates through all active policies and evaluates their status (active, not yet active, or expired). It then determines whether the new data point qualifies as a bad-weather day for the relevant region. If so, the policy is updated accordingly, and the appropriate events are emitted.

Once a policy has expired, the function checks whether the number of bad-weather days exceeds the deductible. If the threshold is met, the payout is computed as
\[
(\text{bad-weather days} - \text{deductible days}) \cdot \text{coverage}.
\]
The contract’s balance is verified to ensure sufficient funds for the payout. If the payment is successfully processed, the expired policy is removed from the contract state, while all associated event logs remain stored on the blockchain.

Figure~\ref{fig:flowchart} provides a visual representation of the \texttt{trigger} function.

\begin{figure}[h]
    \centering
    \includegraphics[width=0.6\textwidth]{flowchart_2}
    \caption{Flowchart of the \texttt{trigger} function}
    \label{fig:flowchart}
\end{figure}
\newpage

%%%%%%%%%%%%%%%%%%%%%%%%%%%%%%%%%%%%%%%%%%%%%%%%%%%%%%%%%
%%%%%%%%%%%%%%%%%%%%%%%%%%%%%%%%%%%%%%%%%%%%%%%%%%%%%%%%%


\section{Conclusion}
\newpage

%%%%%%%%%%%%%%%%%%%%%%%%%%%%%%%%%%%%%%%%%%%%%%%%%%%%%%%%%
%%%%%%%%%%%%%%%%%%%%%%%%%%%%%%%%%%%%%%%%%%%%%%%%%%%%%%%%%

\section*{Appendix}
\addcontentsline{toc}{section}{Appendix}

\newpage

%%%%%%%%%%%%%%%%%%%%%%%%%%%%%%%%%%%%%%%%%%%%%%%%%%%%%%%%%
%%%%%%%%%%%%%%%%%%%%%%%%%%%%%%%%%%%%%%%%%%%%%%%%%%%%%%%%%


\begingroup
\addcontentsline{toc}{section}{References}
\renewcommand*\refname{References}

\begin{thebibliography}{----}

\bibitem{Mack1993}
\textsc{Mack}, T. (1993): {\it Distribution-free calculation of the standard error of chain ladder reserve estimates}. ASTIN Bulletin 23/2, S. 171-183.

\bibitem{LinKwon}
\textsc{Lin}, X., \textsc{Kwon}, WJ. (2020): {\it Application of parametric insurance in principle‐compliant and innovative ways}. Risk Manag Insur Rev 23, p. 121-150.


\bibitem{ParamGuide}
Swiss Re Corporate Solutions,
\textit{Comprehensive guide to Parametric Insurance},
\url{https://corporatesolutions.swissre.com/dam/jcr:0cd24f12-ebfb-425a-ab42-0187c241bf4a/2023-01-corso-guide-of-parametric-insurance.pdf}. Accessed, November 8, 2025.

\bibitem{SwissReParamIns}
Swiss Re Insights,
\textit{Parametric insurance – a long history, a bright future},
\url{https://corporatesolutions.swissre.com/insights/knowledge/evolution-of-parametric-insurance.html}. Accessed, November 8, 2025.

\bibitem{AllianzReport}
Allianz,
\textit{Allianz Global Insurance Report 2025: Rising demand for protection},
\url{https://www.allianz.com/en/economic_research/insights/publications/specials_fmo/250527-global-insurance-report.html}. Accessed, November 8, 2025.

\bibitem{meteostat}
Meteostat,
\textit{Meteostat webiste},
\url{https://meteostat.net/de/}. Accessed, November 10, 2025.

\bibitem{dwd}
Deutscher Wetterdienst,
\textit{Wetterlexikon: Niederschlagsintensität},
\url{https://www.dwd.de/DE/service/lexikon/Functions/glossar.html?lv2=101812&lv3=101906}. Accessed, November 16, 2025.

\bibitem{Szabo1997}
N. Szabo, ``The Idea of Smart Contracts,'' 1997.

\bibitem{Buterin2014}
V. Buterin, ``A Next-Generation Smart Contract and Decentralized Application Platform,'' Ethereum Whitepaper, 2014.

\bibitem{Wood2014}
G. Wood, ``Ethereum: A Secure Decentralised Generalised Transaction Ledger,'' Ethereum Yellow Paper, 2014.

\bibitem{Antonopoulos2018}
A. M. Antonopoulos and G. Wood, \textit{Mastering Ethereum}, O'Reilly Media, 2018.

\end{thebibliography}
\endgroup
\end{document}
